%---------------------------------------------------------------------------
%	Preamble Etc.
%---------------------------------------------------------------------------
%---------------------------------------------------------------------------
%	Packages
%---------------------------------------------------------------------------
\documentclass{article}
\usepackage[english]{babel}
\usepackage[utf8]{inputenc}
\usepackage{fancyhdr}
\usepackage{geometry}
\usepackage{amsmath}
\usepackage{amssymb}
\usepackage{enumitem}
\usepackage{xcolor}
\usepackage{braket}
\usepackage{graphicx}
\usepackage{mathpazo}
\usepackage{mathtools}

\geometry{%
  left=10mm,
  right=10mm,
  top=25mm,
  bottom=25mm,
  bindingoffset=0mm,
  headheight=50pt,% output from geometry tells you what this needs to be set to as a minimum
}

\input{"Files/Commands & Environments.tex"}
%---------------------------------------------------------------------------
%	Document
%---------------------------------------------------------------------------
\begin{document}
\begin{titlepage} 
\newcommand{\HRule}{\rule{\linewidth}{0.5mm}}
\center % Centre everything on the page
%------------------------------------------------
%	Headings
%------------------------------------------------
\includegraphics[width=1.0\textwidth]{OU Physics.png}\\[1cm]
%------------------------------------------------
%	Title
%------------------------------------------------
\HRule\\[0.2cm]
{\huge\bfseries \ClassName}\\[0.0cm] % Title of your document
\HRule\\[0.25cm]
\textsc{\large \underline{\ChapterTitle}}\\[0.5cm] % Minor heading such as course title
%------------------------------------------------
%	Details
%------------------------------------------------
\vspace{50pt}
% Student Name
\begin{minipage}{0.4\textwidth}
	\begin{flushleft}
		\large
		\textsc{Student}\\
		\StudentName % Your name
	\end{flushleft}
\end{minipage}
	~
% Professor Name
\begin{minipage}{0.4\textwidth}
	\begin{flushright}
		\large
		\textsc{Professor}\\
		\ProfessorName % Professor's name
	\end{flushright}
\end{minipage}
%------------------------------------------------
%	Date
%------------------------------------------------
\vfill\vfill\vfill % Position the date 3/4 down the remaining page
%------------------------------------------------
%	Logo
%------------------------------------------------
%\vfill\vfill
\includegraphics[width=0.2\textwidth]{OU Logo.png}\\[1cm]
%----------------------------------------------------------------------------------------
\vfill % Push the date up 1/4 of the remaining page
\end{titlepage}
%---------------------------------------------------------------------------
%	Problem 1
%---------------------------------------------------------------------------
\begin{Problem}{Problem 1:}{
Consider a sphere of radius $R$ and total charge $Q$ that has been embedded with an $r$‐dependent charge density:
\begin{equation*}
\rho(\vec{r}) = Cr
\end{equation*}
\begin{enumerate}[label=(\alph*)]
\item Write and solve an integral to determine $C$ in terms of the properties of the sphere, $Q$ and $r$. For the rest of the problem, use this result for $C$ in $\rho(\vec{r})$.
\item Explain why and how you can use Gauss’ Law to solve for the electric field of the charged sphere.
\item Set up your solution and determine $\vec{E}(\vec{r})$ for all $r$. You will have somewhat different results for $0 \leq r \leq R$ and for $r \geq R$. These results should agree for $r = R$.
\item Solve for $\phi(\vec{r})$ for all $r$. As usual, define $\phi(\infty)=0$.
\item Solve for the total electric potential energy of the charged sphere. This can be considered the total energy (work) necessary to bring all the charges together in the sphere.
\item Consider a similar sphere with the charge density $\rho(\vec{r}) = Cr\cos^{2}{\theta}$. Set up a multipole expansion to solve for $\phi(\vec{r})$ and show (explain) that there will only be two terms in the expansion. If you wish, you can solve this.
\end{enumerate}
}
\end{Problem}
%---------------------------------------------------------------------------
%	Problem 2
%---------------------------------------------------------------------------
\begin{Problem}{Problem 2:}
{
The Magnetic analog to the problem above is a long, current‐carrying wire with a current density that varies across the radius of the wire. \\ \newline
\noindent
Use polar coordinates: $\hat{z}$ along the wire, $\hat{r}$ the radial direction
perpendicular to $\hat{z}$, and $\hat{\phi}$ the azimuthal angle around $\hat{z}$. \\ \newline
\noindent
The wire has a radius $R$ and total current $I$ in the $\hat{z}$ direction. The current density is:
\begin{equation*}
\vec{J}(\vec{r}) = Cr\hat{z}
\end{equation*}
\begin{enumerate}[label=(\alph*)]
\item Write an equation relating the total current in the wire, $I$, to the current density, $\vec{J}(\vec{r})$ for the wire. Use this to Derive an expression for the constant $C$ in terms of properties of the wire, $I$ and $R$.
\item Use the symmetry of the problem and the relation between the current (density) and the magnetic field to determine the general form for the magnetic field, $\vec{B}(\vec{r})$. Specifically, what direction is the field and how does it depend on $r, \phi,$ and $z$?
\item Explain why you can use Ampere's Law to solve for $\vec{B}(\vec{r})$.
\item Calculate $\vec{B}(\vec{r})$ everywhere. Be sure to explain your approach.
\item Show that the magnetic potential energy of the wire is infinite. This shouldn't be surprising, as it is an infinitely long wire.
\end{enumerate}
}
\end{Problem}
%---------------------------------------------------------------------------
%	Problem 3
%---------------------------------------------------------------------------
\begin{Problem}{Problem 3:}
{
Consider a uniform charged rod of charge $Q$ and length $L$, $\lambda = Q/L$, on the $z$-axis, centered at the origin, extending from $z = -\frac{L}{2}$ to $z = \frac{L}{2}$.
\begin{enumerate}[label=(\alph*)]
\item Write an integral over the charged rod for the electric potential on the $z$‐axis. Calculate the electric potential due to the rod on the $z$‐axis for all $z > \frac{L}{2}$.
\item Expand your result for $\phi(z)$ in powers of $L/(2z)$. It will be useful to know that: (If both expressions don't help you, you'll probably want to redo your integral from part (a).)
\begin{equation*}
\ln{(1-x)} = - \sum^{\infty}_{n=1} \frac{x^{n}}{n}, \hspace{10pt} \ln{(1+x)} = \sum^{\infty}_{n=1} (-1)^{n-1} \frac{x^{n}}{n}
\end{equation*}
\item Using this expansion, you can determine a sum for the potential $\phi(r,\theta)$ everywhere. Note that for points along the $z$-axis:
\begin{equation*}
\phi(z) = \phi(r,\theta = 0)
\end{equation*}
The potential everywhere is given by:
\begin{equation*}
\phi(r,\theta) = \sum^{\infty}_{l=0} \frac{a_{l}}{r^{l+1}} P_{l}(\cos{\theta})
\end{equation*}
Using the result from part (b), determine an expression for all the coefficients $a_{l}$.
Note: $P_{l}(\cos(0)) = P_{l}(1) = 1$ for all $l$.
\item Using the multipole expansion for the electric potential, calculate the monopole, dipole, and quadrupole potential terms due to the line charge. Show that these terms agree with part (c).
\end{enumerate}
}
\end{Problem}
%---------------------------------------------------------------------------
%	Problem 4
%---------------------------------------------------------------------------
\begin{Problem}{Problem 4:}
{
A square current‐carrying loop with sides of length $l$ and counter‐clockwise current $I$ is in the $x-y$ plane and centered at the origin. Calculate the magnetic field at the point $\vec{r} = d \hat{x}$.
\begin{enumerate}[label=(\alph*)]
\item What is the magnetic field at the point $\vec{r} = d \hat{x}$ if you approximate the loop by a magnetic dipole?
\item Use cross products to determine the direction of the magnetic field due to each side of the loop at $\vec{r}$.
\item Write down integrals for the magnetic field at $\vec{r}$ due to each side of the current loop.
\item Solve for the magnetic field due to the loop at $\vec{r}$. It might be useful to know that:
\begin{equation*}
\int \frac{dx}{(a^{2}+x^{2})^{3/2}} = \frac{x}{a^{2}\sqrt{a^{2}+x^{2}}}
\end{equation*}
Note: The results are somewhat messy.
\item Expand your result above for $d >> l$ $\frac{l}{d} << 1$ to the lowest non‐zero power in $1/d$. Show that this gives the dipole approximation from part (a).
\end{enumerate}
Hint: It’s a good idea to expand the fields from the right and left wires together, and the top and bottom wires together.
}
\end{Problem}
\end{document}